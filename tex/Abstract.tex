\section*{Abstract}
Thesis \textbf{Using Semantic Analysis of Moldovan Online Media as a Way to Reveal Patterns and Trends} presented by Terman Sergiu was written in English. It has 25 figures, 11 listing, 8 tables, and 14 references. The report consists of introduction 4 chapter and conclusion.

The thesis aims to research the potential of data journalism in Republic of Moldova. For that purpose OpenMedia project was developed. It is a platform which aggregates online media channels for offering means of visualization of word frequency requested by a user.

The application was build using Ruby programming language. The platform consists from two components. The first one is the data gathering and preprocessing. It can be divided in four smaller parts. Crawling the media channels an fetching the article pages, parsing the articles and storing to database, executing natural language processing operations over articles and preparing data for client side visualization. The second part of the platform is the client side application. It aims to provide an interactive way for querying words and visualizing the data. The data representation is done using enhanced line plots that denote the word frequency for different media sources.

The technologies used are Sinatra framework for building the web interface, MongoDB for storing the data and running native mapreduce tasks, Sidekiq for launching asynchronous jobs and Racai SOAP service for NLP operations.

OpenMedia represents an attempt to offer a tool for data journalism targeted for Republic of Moldova. It does not have many features but it has the right infrastructure for future development. It also proved that a data journalism software has a lot of potential on inexistent local market.