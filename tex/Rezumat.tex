\section*{Rezumat}
Lucrarea \textbf{Utizitarea analizei semantice a online media din Moldova, pentru detectarea tendințelor si modelelor} prezentată de Terman Sergiu a fost scrisă în engleză. Ea constă din 25 de figuri, 11 secvențe de cod, 8 tabele și 15 referințe. Lucrarea face parte din introducere, 4 capitole și concluzii.

Teza are ca scop cercetarea potențialului jurnalismului pe bază de date în Republica Moldova. Prin urmare proiectul OpenMedia a fost dezvoltat. Aceasta este o platformă care agreghează canalele media disponibile online, pentru a oferi mijloace de vizualizare a frecvenței menționării a unui cuvânt specificat de către utilizator.

Aplicația a fost dezvoltată utilizând limbajul de programare Ruby. Platforma consită din 2 părți. Prima parte reprezinta adunarea și preprocesarea datelor. Aceasta poate fi divizată în 4 entități. Accesarea canalelor media și descărcarea articolelor, parsarea fișierelo descărcate pentru a extrage conținutul textual al articolelor și salvarea lor în bază de date, aplicarea operațiilor de procesare a limbajului natural, pregătirea datelor pentru a fi vizualizate. A 2 parte a platformei reprezintă aplicația utilizatorului. Aceasta are drept obiectiv să ofere o modalitate interactivă de interogare a cuvintelor și vizualizare a datelor. Reprezentarea datelor este realizată prin intermediul graficilor liniare ce denotă frecvența cuvântului în cadrul diferitor surse media.

Tecnologiile utilizate sunt următoarele. Sinatra framework pentru dezvoltarea interfaței web. Baza de date folosită este MongoDB. Aplicația Sidekiq este folosită pentru lansarea sarcinilor asincrone. Mai este folosit și serviciul web Racai, pentru executarea operațiunilor de procesare a limbajului natural.


OpenMedia reprezintă o încercare de a oferi un instrument de jurnalism pe bază de date orientată pentru Republica Moldova. Ea nu cuprinde multe funcționalități însă are infrastructura necesară pentru o dezvoltare în viitor. Acest proiect a demonstat faptul că așa un instrument are un mare potențial pe piață locală, fiind pioner în acest domeniu.
