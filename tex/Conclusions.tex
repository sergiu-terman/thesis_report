\section*{Conclusions}
\phantomsection
Data science is a fresh evolving field that has a positive impact over more that a single economical area. The use cases of its applicability are so various that it resulted in many other narrow subfields. Data science aims to extract meaningful information out of tremendous amounts of data. The data sources are various and every field has its own origin. Data journalism is a surfacing domain of data science that aims to facilitate the classic journalism by providing a set of tools that offer diverse way of visualization data under different formats. Such kind of tools targets journalists that can write quality stories. The distinguishing key between the classic and data journalism is that the later one opens the doors to writing the most compelling articles. It can be used to connect dots and prove theories that can give as a result meaningful statements backed up by viable sources.

OpenMedia project is the outcome of the thesis work. It is a platform that aggregates online media sources from Republic of Moldova in order to offer basic tools for data visualization. From a big picture the application represents a data warehouse system. The uniqueness of the platform is that the targeted users and the data sources is unusual for the moldavian market. There still are no tools available that does a similar thing on the local level. Developing a software that doesn't yet exist on a market offers a lot of freedom, but at the same time is problematic because the needs of the potential customers are yet blurry. A joint work with a journalism representative would have resulted into a more specific and useful product.

Building the application required many multi-step planning of the infrastructure. The problem encountered with collecting the data is that some media sources does not encode the article URL under a specific pattern that can be followed. Which means that some media sources are more easy to integrate and some not. A last resort solution is to directly contact the representatives of media source and convince them to give data access. This should not represent a problem because the articles are publicly available on the web sites hosted by the media channels.

Another major problem encountered while developing the platform is the natural language processing operations. Unfortunately, for the Romanian language, there aren't any public available libraries that can be used for applying NLP operations. The only founded solutions were SOAP web services provided by two sources, more specifically two universities from Romania. The problem is that using a web service for processing huge amount of textual data takes a lot of time and leaves no room for experiments. Another problem is the debugging process is tedious, tinkering with SOAP messages in order to understand the errors is an exhaustive process. The NLP part of the applications relies heavily on these two sources. In case if the web services are not available anymore the platform will not be able to provide further results for new incoming data. This concludes that OpenMedia has strong dependencies with the NLP web services.

There are lots of ideas for further development of the project. In order to make them more flourishing it is be better establish a collaboration with a media representative. The first element that would be added are the basic registration, profile routine. Every user will have a personal query history that will allow to browse the previous results faster. Another important part is adding more visualization tools. In this case the media corespondent might give suggestions for building a more specific requirements. In the immediate future is planned to implement the trend detection feature. Another essential part is to integrate as many media channels as possible, it is crucial to offer a result coming from multiple sources. In a long run it is intended is to create emotional analysis upon articles but this kind of operations requires a lot of resources and research.

A quintessential part for the future development of OpenMedia is to remove the dependencies with the web services that provides the NLP operations. The services are useful for the first phase of development but when it comes to building reliable applications it can not be afforded to rely on external tools. It might put the whole platform in jeopardy. Building a trustworthy NLP library would require a lot of time investment and research but the result is worthwhile.

Another future implementation is to reconstruct the client side application from the perspective of creating a better user experience. It would also require a designers touch, which the application totally lacks. Also a lot of time will be invested for building tools for data visualization. Just displaying results on the screen is not good enough if it doesn't communicate a message to the user. OpenMedia was designed to help journalists and not only, to create worthwhile and compelling works.
\clearpage