\section*{Introduction}
\pagenumbering{arabic}
\setcounter{page}{12}
\phantomsection
An average person daily consumes a continuous stream of information. It is represented in different forms under various formats. And due to the great amounts of data that surrounds us nowadays, communication efficiency became a crucial problem. Throughout the time, data representation suffered a sequence of changes, thus constantly evolving. Starting from architecture, letters on papers ending with radio, television and computers wired through Internet.

Textual format remains the most common method of transmitting information. Though not the most efficient, it is cheap in terms of storage, expressive and extensive. The problem is that people encounter problems in the process of absorbing textual data. The mnemonic techniques usually are not enough for extracting the key information from text.

Text processing on a bigger scale is a tedious operation for a human being. That's why people resort to computers in order to solve the problem more efficiently. Data Science, a fairly new field in computer science, tackles this problem. It employs techniques and theories drawn from many fields within the broad areas of mathematics, statistics, information theory, probability models, machine learning, data mining, database, pattern recognition and many others. The problems that it solves are marketing optimization, fraud detection, trends detection even sport bets.

The problem researched in this work is regarding online media, articles published by Moldavian sources such as Timpul, Unimedia, Publika. The goal is to create a platform where users could query word frequency, search for the trends relevant for different periods of time, observe the sources that are emphasizing specific key words, detect biases. The purpose is to provide visualization methods for drawing viable conclusions. Why it is important to have such tool on the local market? Because there isn't one present. People running business, from various areas, could tune their business model approach based on results provided by the platform. Journalism is another niche targeted by the platform. It is not easy to write an article regarding events that happened in the course of an year. It is much simpler when a journalist can observe things like: when was a specific topic more popular or when it started to become less known. Which political entity is more discussed, in what period of time? News media is the quintessential origin of information feed. The good thing is that it can be processed and used, in terms of software, for providing data in an understandable manner.

The application would consist from two separate parts. The first one represents the data mining platform. It has the purpose to go through media web portals and pull all the available articles, followed by extracting the articles content and storing into a database. The next step is running natural language processing tasks in order to enrich the articles with useful metadata, that are later used for meaningful queries and proper data visualization. The second part represents the client interaction application. It is a web platform that provides querying features and visualization options. The queries most likely will represent mapreduce tasks that will run on the same storage used by the first platform. The communication bridge between the application components is the database. The client part of application will have a notifications systems.

\clearpage