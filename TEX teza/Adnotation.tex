\section*{Adnotation}

Thesis {\bf Mathematical models and methods for improving the conversion efficiency of renewable energy based on aero-hydrodynamics effects}, presented by Viorel Bostan for the competition of doctor habilitate degree in techincal sciences, was developed at the Technical University of Moldova , Chisinau, is written in Romanian and contains 433 pages, 145 figures, 8 tables, and 223 references. The thesis consists of introduction, six chapters, conclusions and appendices. Appendicies contain additional 57 figures and 49 tables.

The thesis is dedicated to the study of aero-hydrodynamic effects in small power ($P<20$ kW) wind turbine rotors and micro hydropower stations rotors using mathematical models to describe turbulent flows and modern methods of numerical simulation in the framework of computational fluid dynamics (CFD).

The purpose of this work is to increase the conversion efficiency and functional capacity of small power wind turbines and micro hydropower stations.

There were identified models and modern mathematical methods used to describe the turbulent flows specific to small power rotors with emphasis on aero-hydrodynamic unsteady effects and near blades effects. There was argued the geometry of efficient aero-hydrodynamics blades  in terms of energy conversion efficiency based on which original concepts of aero-hydrodynamic rotors were developed.

Using the proposed CAD rotor models there were performed complex  CFD simulations of the flow through the rotors and near the blades  in order to determine the influence of the constructive and  kinematic parameters on the power and performance factor characteristics of  of the aero-hydrodynamic rotors used in wind turbines and micro hydropower stations; there were performed the boundary layer analyzis and identified the technical solutions that can assure the decrease of possible negative effects on the energy conversion efficiency.

Based on the obtained research results, there have been developed and manufactured new models of small power wind turbines and micro hydropower stations for various applications, including the concept of a wind turbine with tilting rotor and orientation to the wind direction through windrose-wheels. The developed technical solutions were protected by 17 patents and appreciated at International Innovations, Research and Technology Transfer Salons  with 43 ​​gold, 13 silver and 2 bronze medals.

Keywords: mathematical modeling; CFD numerical simulation; boundary layer; turbulent flow; aero-hydrodynamic rotor; wind turbine; small hydropower station.
