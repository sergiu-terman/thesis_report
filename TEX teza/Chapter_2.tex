\section{Software Design}
\phantomsection

This chapter contains the description of the software development process
used to bulid the application. We start with the description of the
iterative and incremental model of developing software and justify its usage
for this project. In the spirit of the chosen methodology we describe the
major iterations of developing the application. For each iteration we state
what subset of the initial requirements was implemented and state them more
formally as a use case scenario that are used to get more insight into what
should be implemented and verify if the designed module fullfills the
outcomes set at the start of the iteration. We continue by describing the
implementation details and presenting portions of source code in this
context. In the last section we present a general overview of modular
representation of the application and what are the connections between
various application components.

\subsection{The Software Development Process}

In order to develop the application for this thesis project there was chosen the
Iterative and Incremental approach. Iterative development is an approach to
building software in which the overall lifecycle is composed of several
iterations in sequence \cite{Larman1}. 
The iterations are organized into a series of short, usually fixed length
steps called iterations, each iteration providing a tested implementation of
a part of the system. The system is successively enlarged and refined during
the iterations. An important input into each iteration is the feedback
received from the client and the readaptation of the system to the changing
requirements as the user understands and refines his view upon what
functionality the software should deliver. The process is illustrated in
figure \ref{IterativeAndIncremental}

\begin{figure}[!ht]
\centering
\includegraphics[width=13cm]{iter}
\caption{Iterative and Incremental Development, \cite{Larman2}}\label{IterativeAndIncremental}
\end{figure}

The output of each iteration is not an experimental prototype but a ready to
use subset of the final system. During each iteration a subset of the
software requirements is selected and implemented. However an interation
shouldn't necessarily include the development of a new part of the system.
Rather it could include some implementation steps for improving the
performance of the existing modules.

The reason to use this approach for developing the application is that the
application itself as well as the domain in which it will be used are
relatively new. There are no current applications on the market suited for
the needs of the project simply because it is a part of a research project
implementing an innovative idea in the field of renewable energy sources.
Therefore there are no clear specifications of the software that can be
taken as initial requirements. Also the approach itself allows the
researchers to view intermediate versions of the software and refine their
views upon how the workflow should be implemented, how the UI\ should look,
etc.

\subsection{Software Requirements}

Although the application is built in small increments during successive
iterations, a vision about the software under development and the list of
requirements must be established at the beginning of the project. The
description should contain the information about what kind of application is
being developed (desktop, client-server, etc), who will be the final users
and what are their goals, what domain processes or activities should the
software implement. However at this stage it is impossible to define use
case scenarios in detail. The main outcome of this step should be a list of
system requirements prioritized by the value they have for the end user.

MHSim is a desktop application that will be used as a modeling and
computational tool for the design and analysis of the working regime of the
power station's rotor with rotating blades. The system will allow the user
to specify the properties of the flow, the rotor and its blades. From a
computational perspective the system must compute and render the plots for
functional characteristics such as the forces which act on a blade, the
torque produced by one blade, the resultant force and torque produced by all
blades and the information about the power station floating stability. As a
modeling tool the application should provide an ergonomic and intuitive UI,
that is the structural parameters should be properly grouped and the UI
should structure the steps of the workflow in the same way as if the
modeling activity would be done in a real environment. Because the
computational routines might last for a considerable amount of time, and the
solution isn't guaranteed to converge for each case, the user must be able
to control its execution. The application will be used by researchers to
analyze a given configuration of the rotor with rotating blades and select
the model which provides the best performance characteristics based on the
data generated by the program. Since building the model of the rotor is only
a stage in the overall power station design the application must provide the
functionality of saving the data for further processing.

Next there are listed the software requirements derived from the high-level
description made above and several discussion sessions with the end-user.

\begin{enumerate}[topsep=5pt, partopsep=0pt,itemsep=3pt,parsep=1pt]
\item The system must provide the values of the hydrodynamic coefficients
for a given airfoil profile.
\item The execution of the computational routine should be controlled by the
user. This includes:

  \begin{enumerate}
  \item completion status visualization;
  \item visualizing the computational strategies used;
  \item cancelling the solutions which don't converge.
  \end{enumerate}

\item The user should be able to select an airfoil by specifying a NACA\
profile or reading the coordinates of the airfoil from a file. The number of
discretization points should be customizable.
\item The user must be able to enter the following properties of the fluid:
density, viscosity and the freestream velocity.
\item The user must be able to specify the following parameters of the rotor
and blades:
   \begin{enumerate}
   \item the length of the rotor lever;
   \item the number of rotating blades;
   \item the height and chord length of the blades;
   \item the point where the blade is fixed to the rotor's lever;
   \item the number of regions in which a blade changes its direction relative
           to the flow. For each region the following information should be specified:
       \begin{enumerate}
       \item the angle formed by the sector arms;
       \item the angle between the blade and the rotor lever;
       \item the angle between the blade and the flow at the start and end regions of the sector;
        \item an indicator showing if the blade is carried freely by the water flow (the blade in that sector won't contribute to the total                 
                moment generated by the rotor);
       \end{enumerate}
    \end{enumerate}
\item The system must show the direction of the flow and the direction of rotation of the power station.
\item The system must provide the graphs and numerical data for the following functional characteristics:
   \begin{enumerate}
   \item Hydrodynamic coefficients versus angle of attack;
   \item Lift, drag, tangential and total force versus position angle;
   \item Torque generated by a blade and total torque generated by all blades;
   \item Buoyant stability data: the path traced by the center of application of buoyant forces and the vertical displacement of the center versus position angle.
   \end{enumerate}
\end{enumerate}

The user requirements are organized into the use case\ diagram shown in
figure \ref{UseCases}

\begin{figure}[!ht]
\centering
\includegraphics[width=13cm]{ucSystem}
\caption{System use case diagram}\label{UseCases}
\end{figure}

The sections that follow describe the software development process as a
sequence of iterations illustrating the way that the system was built --
using the incremental and iterative approach. Each iteration (except first
which is mostly an implementation iteration) \ start with a use case
scenario that describes the functionality of the system that is implemented
in that iteration. We analyze the use case scenario to emphasize the details
that affect the design of the implemented module. Then the design solution
is presented along with the discussion of the implementation issues. During
the iteration the usage and implementation of design patterns is emphasized
and a short description of what design problems it mitigates is given.

\subsection{Iteration 1 -- linking to XFOIL code}

During this iteration we develop a module that provides a simplified
interface to the computational routines defined in XFOIL that later are used
by the rest of the application to fullfil the first system requirement - to
provide the values of the hydrodynamic coefficients for a given airfoil
profile. The module provides the functions to generate and retrieve airfoil
coordinates and extract the hydrodynamic coefficients at low angles of
attack. Since this is an implementation step no use case scenarios are
defined for this iteration.

\subsubsection{Modifications to XFOIL code}

The algorithms for determining hydrodynamic coefficients of lift, moment and
drag forces require deep knowledge of CFD and dexterity in implementing
complex computational methods. An example of the description of computation
steps involved are described in \cite{Bostan}. An alternative approach would be to study the existing open source
applications and use the already tested routines. The problem is that none
of the analyzed programs provide such a module as a software library that
can be added to our application. One well known program that allows to
design and analyze isolated airfoils at subsonic speeds is XFOIL \cite{Xfoil}
. It is released under GNU\ General Public License which means that anyone
is allowed to study and modify the source code. The first obvious problem is
that XFOIL is designed as an interactive program: the user is presented with
a command line menu where he enters commands and recieves response for each
of the multiple options available. The other implementation problem is that
XFOIL is written in Fortran 77, an old dialect of Fortran language approved
as a standard in 1978 but the main part of the application will be written
in Java. On the other side Fortran is an imperative language specifically
designed for scientific and other computational intensive applications,
meaning that the generated source code will perform faster than in most
other programming langauges.

After studying the source code for 4 weeks solving the first problem proved
to be a quite manageable task. The only modifications done at the first
iteration involved transforming the conditional loops that accept user input
and match against a set of predefined commands into linear routines. The
input is passed from C++ language binding functions simulating the entered
commands as if typed by the user. Gathering the results of the computations
also require a small amount of programming, but in order to better
understand it, the programming model used in XFOIL\ is shortly described
next.

{\bf XFOIL\ programming model.} Following the classification of programming languages done in \cite{booch}, the programming model used in XFOIL resembles the one used in Late Third-Generation Programming Languages shown in figure \ref{Late3ProgrammingLanguages}.

\begin{figure}[!ht]
\centering
\includegraphics[width=0.5\textwidth]{Late3ProgrammingLang}
\caption{The Topology of Late Third-Generation Programming Languages, \cite{booch}}\label{Late3ProgrammingLanguages}
\end{figure}

The main entry point in XFOIL contains a loop that matches user input
against the option from the top level menu. The selection of some option is
followed either by the direct execution of the computational routines or the
invocation of a lower lever menu that presents another set of the selected
menu specific options. Among others there are 2 types of commands: commands
that directly invoke the routines for implementing the selected option such
as the command to generate a NACA\ airfoil profile or commands which sole
purpose is to set a parameter used in further computations, for example
setting the Reynolds number for the computation of the boundary layer
equations for viscous fluids. The state of the computations performed in
different routines is stored in \textit{common blocks}, a Fortran specific
language construct. Basically a common block represents a named block of
memory that stores a set of variables. Data such as the coordinates of the
airfoil profile or the current Reynolds number are stored in such blocks and
may be accessed as long as the program is loaded into memory. New data may
be freely added in this global storage by defining new common blocks. For
the needs of the current application was added a block to hold the computed
hydrodynamic coefficients for angles of attack between $\pm $20 degrees. In
order to set the data the assignment statements that set the values of these
coefficients must have been inserted in the proper place in the code (the
code is located in EXEC\_OPER\_COMMAND a modified version of OPER subroutine
which implemented a lower-lever menu which among others provides the options
to compute the hydrodynamic coefficients).The same approach was taken in the
next iteration when implementing the customizations for the coefficient
extraction interface.

{\bf Linking Fortran and C++ code.} In order to link the code written in XFOIL with the rest of the application
we need to create C++ language bindings. By bindings we mean a set of
functions that call the functions defined in the other language, extract
data from global storage, etc. In the case of our application the bindings
also must provide a simplified and restricted interface to XFOIL\ code that
would make the whole module more usable by the rest of the application code.

Linking C++ with Fortran code required the knowledge of the following
typical mixed-language programming topics:

\begin{enumerate}[topsep=5pt, partopsep=0pt,itemsep=3pt,parsep=1pt]
\item Name mangling;
\item Calling conventions used in both languages;
\item Specifics of parameter passing (is the parameter passed by value or
reference, is the length of the array passed implicitly or not);
\item Primitive data type mappings (how similar data types are represented
in memory, how are represented language specific data types, e.g. LOGICAL
data type in Fortran);
\item How to represent language specific constructs (e.g. how to represent
common blocks in C++ language).
\end{enumerate}

If these aspects are not studied and handled properly the end-result usually
manifests itself as a severe run-time error. A standard Fortran subroutine
passes the parameters by address and has \textit{\_\_stdcall} calling
convention. Name mangling can be handled by specifying additional compiler
options or by declaring (in the binding code) the called functions with
additional decoration symbols such as underscores (unless the mangling
scheme is complex or results in invalid identifiers). A\ common block in
Fortran maps to the declaration of a data structure in C++ with the same
order of variables as declared in Fortran code, the common block itself is
represented by an instance of the data structure with the same name as the
common block in Fortran code. The listings \ref{list3} and \ref{list4} illustrate this
mapping.

\lstinputlisting[language=Fortran, caption={Common block declaration in Fortran}, label=list3]{../SrcCode/commonBlock.f}

\lstinputlisting[language=C, caption={Common block declaration in C++}, label=list4]{../SrcCode/commonBlock.c}

\subsubsection{C++ language bindings for XFOIL\ code}

The class diagram shown in figure \ref{clsBindings1} shows the structural relationships between the concepts defined in the code
that links to XFOIL program. In order to model C functions and structures
using UML the approach described in \cite{CwithUML} is used. The idea is to model a file containing C language code as a class
stereotyped with $<<$File$<<$. As additional information we added a tagged value indicating in which
language the file is written (on the diagram we have files written in C++
and Fortran languages). The functions and variables defined in a file
represent the attributes and methods of the class. The visibility of
attributes and methods can be either public or private. Public members
represent code and data that is visible outside the file, that is functions
and global variables declared in header files. On the other hand private
members are functions and data inaccessible outside the file (functions and
variables declared with \textit{static} modifier). The simplified interface
to XFOIL code is represented by the set of public methods in the \textit{%
exports} class. The relationships between classes are mostly dependencies
stereotyped as $<<$call$>>$ and are used to indicate the files that contain functions that depend on
functions defined in other files. We can also see associations going from 
\textit{xfoil\_data\_structures} to C++ data structures declared in that
file. The structures represent the C++ language equivalent of common blocks
in Fortran from which functions defined in \textit{exports} later will
extract data such as the coordinates of the airfoil, the values of the
hydrodynamic coefficients, etc. or set the data in those common blocks. On
the diagram are shown only the fields from the common blocks which are used
by the binding module. Next will be described the usage of each common block
in the context of our application:

\begin{enumerate}[topsep=5pt, partopsep=0pt,itemsep=3pt,parsep=1pt]
\item CI01 - stores the length of the airfoil identifier string;

\item CC01 - stores the airfoil identifier;

\item CR05 - stores the airfoil coordinates;

\item CL01 - stores flags set by XFOIL code (\textit{lvisc} - used to
determine if the viscous solution should be computed, \textit{lnorm} - if
the airfoil should be normalized [its chord must have unit length], \textit{%
lvconv} - if the viscous solution converged for a given angle of attack);

\item CI04 - stores various integer variables among which the number of
points on the current airfoil (\textit{n}), maximal number of Netwon
iterations to be performed before convergence is not attained (\textit{itmax}%
) and the maximal number of points for which the solution may not converge (%
\textit{nseqex});

\item MYDATA - the common block added to XFOIL code that is used to store
the hydrodynamic coefficients for angles of attack between $\pm $20 degrees.
The block also contains an indicator set by XFOIL code to signal a file
error in the routine for the extraction of airfoil coordinates from a file.
\end{enumerate}

To illustrate how the interface simplifies the access to XFOIL\ code lets
show an example of an actual sequence of function calls. The sequential
diagram is shown in figure \ref{sqCbindings1}. The diagram illustrates how the client module sets and gets the
coordinates of an airfoil. The client calls the method \textit{setAirfoil()}
to generate/read the coordinates of the airfoil and then store them in 
\textit{CR05} block. The airfoil specifier could be either a file name or a
NACA\ specifier. It is the responsiblity of the function to determine the
type of specifier and what XFOIL command to generate. The function that
simulates user input is called \textit{xfoil\_exec.} The function takes as
parameters the command name and the argument(s). Because the client could
request an option from 3 different XFOIL\ menus (top level menu, panel
manipulation menu or airfoil analysis menu) we delegate the responsibility
for determing the type of menu to code written in \textit{xfoil\_commands}
file, namely to function \textit{type\_of\_command}. The module contains a
set of arrays that stores the subset of XFOIL commands used by the
application. The returned identifier is used to select the function which
dispatches the command to the appropriate XFOIL menu. Note that passing
strings to Fortran requires passing also the length of the string (which is
done implicitly in Fortran code). If the airfoil wasn't generated (for
example if the file with airfoil coordinates could not be read) the function
returns a failure indicator value, otherwise it continues by sending the
commands to discretize the profile with the number of panels specified by
the client code as a second parameter to \textit{setAirfoil()} function. The
details of dispatching those commands where not included in the diagram
since they generate the same sequence of calls as in the case when first
command was sent. After receiving a success indicator value, the client
could extract the arrays containing the airfoil coordinates by calling 
\textit{getAirfoil()} function. The coordinates of the airfoil are extracted
from \textit{CR05} common block.

\begin{sidewaysfigure}[!ht]
\centering
\includegraphics[scale=0.45]{clsBindings1}
\caption{C++ language bindings -- static view}\label{clsBindings1}
\end{sidewaysfigure}

\clearpage

\begin{sidewaysfigure}[!ht]
\centering
\includegraphics[scale=0.75]{sqCbindings1}
\caption{Setting and extracting airfoil coordinates}\label{sqCbindings1}
\end{sidewaysfigure}

\clearpage